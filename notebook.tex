
% Default to the notebook output style

    


% Inherit from the specified cell style.




    
\documentclass[11pt]{article}

    
    
    \usepackage[T1]{fontenc}
    % Nicer default font (+ math font) than Computer Modern for most use cases
    \usepackage{mathpazo}

    % Basic figure setup, for now with no caption control since it's done
    % automatically by Pandoc (which extracts ![](path) syntax from Markdown).
    \usepackage{graphicx}
    % We will generate all images so they have a width \maxwidth. This means
    % that they will get their normal width if they fit onto the page, but
    % are scaled down if they would overflow the margins.
    \makeatletter
    \def\maxwidth{\ifdim\Gin@nat@width>\linewidth\linewidth
    \else\Gin@nat@width\fi}
    \makeatother
    \let\Oldincludegraphics\includegraphics
    % Set max figure width to be 80% of text width, for now hardcoded.
    \renewcommand{\includegraphics}[1]{\Oldincludegraphics[width=.8\maxwidth]{#1}}
    % Ensure that by default, figures have no caption (until we provide a
    % proper Figure object with a Caption API and a way to capture that
    % in the conversion process - todo).
    \usepackage{caption}
    \DeclareCaptionLabelFormat{nolabel}{}
    \captionsetup{labelformat=nolabel}

    \usepackage{adjustbox} % Used to constrain images to a maximum size 
    \usepackage{xcolor} % Allow colors to be defined
    \usepackage{enumerate} % Needed for markdown enumerations to work
    \usepackage{geometry} % Used to adjust the document margins
    \usepackage{amsmath} % Equations
    \usepackage{amssymb} % Equations
    \usepackage{textcomp} % defines textquotesingle
    % Hack from http://tex.stackexchange.com/a/47451/13684:
    \AtBeginDocument{%
        \def\PYZsq{\textquotesingle}% Upright quotes in Pygmentized code
    }
    \usepackage{upquote} % Upright quotes for verbatim code
    \usepackage{eurosym} % defines \euro
    \usepackage[mathletters]{ucs} % Extended unicode (utf-8) support
    \usepackage[utf8x]{inputenc} % Allow utf-8 characters in the tex document
    \usepackage{fancyvrb} % verbatim replacement that allows latex
    \usepackage{grffile} % extends the file name processing of package graphics 
                         % to support a larger range 
    % The hyperref package gives us a pdf with properly built
    % internal navigation ('pdf bookmarks' for the table of contents,
    % internal cross-reference links, web links for URLs, etc.)
    \usepackage{hyperref}
    \usepackage{longtable} % longtable support required by pandoc >1.10
    \usepackage{booktabs}  % table support for pandoc > 1.12.2
    \usepackage[inline]{enumitem} % IRkernel/repr support (it uses the enumerate* environment)
    \usepackage[normalem]{ulem} % ulem is needed to support strikethroughs (\sout)
                                % normalem makes italics be italics, not underlines
    

    
    
    % Colors for the hyperref package
    \definecolor{urlcolor}{rgb}{0,.145,.698}
    \definecolor{linkcolor}{rgb}{.71,0.21,0.01}
    \definecolor{citecolor}{rgb}{.12,.54,.11}

    % ANSI colors
    \definecolor{ansi-black}{HTML}{3E424D}
    \definecolor{ansi-black-intense}{HTML}{282C36}
    \definecolor{ansi-red}{HTML}{E75C58}
    \definecolor{ansi-red-intense}{HTML}{B22B31}
    \definecolor{ansi-green}{HTML}{00A250}
    \definecolor{ansi-green-intense}{HTML}{007427}
    \definecolor{ansi-yellow}{HTML}{DDB62B}
    \definecolor{ansi-yellow-intense}{HTML}{B27D12}
    \definecolor{ansi-blue}{HTML}{208FFB}
    \definecolor{ansi-blue-intense}{HTML}{0065CA}
    \definecolor{ansi-magenta}{HTML}{D160C4}
    \definecolor{ansi-magenta-intense}{HTML}{A03196}
    \definecolor{ansi-cyan}{HTML}{60C6C8}
    \definecolor{ansi-cyan-intense}{HTML}{258F8F}
    \definecolor{ansi-white}{HTML}{C5C1B4}
    \definecolor{ansi-white-intense}{HTML}{A1A6B2}

    % commands and environments needed by pandoc snippets
    % extracted from the output of `pandoc -s`
    \providecommand{\tightlist}{%
      \setlength{\itemsep}{0pt}\setlength{\parskip}{0pt}}
    \DefineVerbatimEnvironment{Highlighting}{Verbatim}{commandchars=\\\{\}}
    % Add ',fontsize=\small' for more characters per line
    \newenvironment{Shaded}{}{}
    \newcommand{\KeywordTok}[1]{\textcolor[rgb]{0.00,0.44,0.13}{\textbf{{#1}}}}
    \newcommand{\DataTypeTok}[1]{\textcolor[rgb]{0.56,0.13,0.00}{{#1}}}
    \newcommand{\DecValTok}[1]{\textcolor[rgb]{0.25,0.63,0.44}{{#1}}}
    \newcommand{\BaseNTok}[1]{\textcolor[rgb]{0.25,0.63,0.44}{{#1}}}
    \newcommand{\FloatTok}[1]{\textcolor[rgb]{0.25,0.63,0.44}{{#1}}}
    \newcommand{\CharTok}[1]{\textcolor[rgb]{0.25,0.44,0.63}{{#1}}}
    \newcommand{\StringTok}[1]{\textcolor[rgb]{0.25,0.44,0.63}{{#1}}}
    \newcommand{\CommentTok}[1]{\textcolor[rgb]{0.38,0.63,0.69}{\textit{{#1}}}}
    \newcommand{\OtherTok}[1]{\textcolor[rgb]{0.00,0.44,0.13}{{#1}}}
    \newcommand{\AlertTok}[1]{\textcolor[rgb]{1.00,0.00,0.00}{\textbf{{#1}}}}
    \newcommand{\FunctionTok}[1]{\textcolor[rgb]{0.02,0.16,0.49}{{#1}}}
    \newcommand{\RegionMarkerTok}[1]{{#1}}
    \newcommand{\ErrorTok}[1]{\textcolor[rgb]{1.00,0.00,0.00}{\textbf{{#1}}}}
    \newcommand{\NormalTok}[1]{{#1}}
    
    % Additional commands for more recent versions of Pandoc
    \newcommand{\ConstantTok}[1]{\textcolor[rgb]{0.53,0.00,0.00}{{#1}}}
    \newcommand{\SpecialCharTok}[1]{\textcolor[rgb]{0.25,0.44,0.63}{{#1}}}
    \newcommand{\VerbatimStringTok}[1]{\textcolor[rgb]{0.25,0.44,0.63}{{#1}}}
    \newcommand{\SpecialStringTok}[1]{\textcolor[rgb]{0.73,0.40,0.53}{{#1}}}
    \newcommand{\ImportTok}[1]{{#1}}
    \newcommand{\DocumentationTok}[1]{\textcolor[rgb]{0.73,0.13,0.13}{\textit{{#1}}}}
    \newcommand{\AnnotationTok}[1]{\textcolor[rgb]{0.38,0.63,0.69}{\textbf{\textit{{#1}}}}}
    \newcommand{\CommentVarTok}[1]{\textcolor[rgb]{0.38,0.63,0.69}{\textbf{\textit{{#1}}}}}
    \newcommand{\VariableTok}[1]{\textcolor[rgb]{0.10,0.09,0.49}{{#1}}}
    \newcommand{\ControlFlowTok}[1]{\textcolor[rgb]{0.00,0.44,0.13}{\textbf{{#1}}}}
    \newcommand{\OperatorTok}[1]{\textcolor[rgb]{0.40,0.40,0.40}{{#1}}}
    \newcommand{\BuiltInTok}[1]{{#1}}
    \newcommand{\ExtensionTok}[1]{{#1}}
    \newcommand{\PreprocessorTok}[1]{\textcolor[rgb]{0.74,0.48,0.00}{{#1}}}
    \newcommand{\AttributeTok}[1]{\textcolor[rgb]{0.49,0.56,0.16}{{#1}}}
    \newcommand{\InformationTok}[1]{\textcolor[rgb]{0.38,0.63,0.69}{\textbf{\textit{{#1}}}}}
    \newcommand{\WarningTok}[1]{\textcolor[rgb]{0.38,0.63,0.69}{\textbf{\textit{{#1}}}}}
    
    
    % Define a nice break command that doesn't care if a line doesn't already
    % exist.
    \def\br{\hspace*{\fill} \\* }
    % Math Jax compatability definitions
    \def\gt{>}
    \def\lt{<}
    % Document parameters
    \title{Equations}
    
    
    

    % Pygments definitions
    
\makeatletter
\def\PY@reset{\let\PY@it=\relax \let\PY@bf=\relax%
    \let\PY@ul=\relax \let\PY@tc=\relax%
    \let\PY@bc=\relax \let\PY@ff=\relax}
\def\PY@tok#1{\csname PY@tok@#1\endcsname}
\def\PY@toks#1+{\ifx\relax#1\empty\else%
    \PY@tok{#1}\expandafter\PY@toks\fi}
\def\PY@do#1{\PY@bc{\PY@tc{\PY@ul{%
    \PY@it{\PY@bf{\PY@ff{#1}}}}}}}
\def\PY#1#2{\PY@reset\PY@toks#1+\relax+\PY@do{#2}}

\expandafter\def\csname PY@tok@w\endcsname{\def\PY@tc##1{\textcolor[rgb]{0.73,0.73,0.73}{##1}}}
\expandafter\def\csname PY@tok@c\endcsname{\let\PY@it=\textit\def\PY@tc##1{\textcolor[rgb]{0.25,0.50,0.50}{##1}}}
\expandafter\def\csname PY@tok@cp\endcsname{\def\PY@tc##1{\textcolor[rgb]{0.74,0.48,0.00}{##1}}}
\expandafter\def\csname PY@tok@k\endcsname{\let\PY@bf=\textbf\def\PY@tc##1{\textcolor[rgb]{0.00,0.50,0.00}{##1}}}
\expandafter\def\csname PY@tok@kp\endcsname{\def\PY@tc##1{\textcolor[rgb]{0.00,0.50,0.00}{##1}}}
\expandafter\def\csname PY@tok@kt\endcsname{\def\PY@tc##1{\textcolor[rgb]{0.69,0.00,0.25}{##1}}}
\expandafter\def\csname PY@tok@o\endcsname{\def\PY@tc##1{\textcolor[rgb]{0.40,0.40,0.40}{##1}}}
\expandafter\def\csname PY@tok@ow\endcsname{\let\PY@bf=\textbf\def\PY@tc##1{\textcolor[rgb]{0.67,0.13,1.00}{##1}}}
\expandafter\def\csname PY@tok@nb\endcsname{\def\PY@tc##1{\textcolor[rgb]{0.00,0.50,0.00}{##1}}}
\expandafter\def\csname PY@tok@nf\endcsname{\def\PY@tc##1{\textcolor[rgb]{0.00,0.00,1.00}{##1}}}
\expandafter\def\csname PY@tok@nc\endcsname{\let\PY@bf=\textbf\def\PY@tc##1{\textcolor[rgb]{0.00,0.00,1.00}{##1}}}
\expandafter\def\csname PY@tok@nn\endcsname{\let\PY@bf=\textbf\def\PY@tc##1{\textcolor[rgb]{0.00,0.00,1.00}{##1}}}
\expandafter\def\csname PY@tok@ne\endcsname{\let\PY@bf=\textbf\def\PY@tc##1{\textcolor[rgb]{0.82,0.25,0.23}{##1}}}
\expandafter\def\csname PY@tok@nv\endcsname{\def\PY@tc##1{\textcolor[rgb]{0.10,0.09,0.49}{##1}}}
\expandafter\def\csname PY@tok@no\endcsname{\def\PY@tc##1{\textcolor[rgb]{0.53,0.00,0.00}{##1}}}
\expandafter\def\csname PY@tok@nl\endcsname{\def\PY@tc##1{\textcolor[rgb]{0.63,0.63,0.00}{##1}}}
\expandafter\def\csname PY@tok@ni\endcsname{\let\PY@bf=\textbf\def\PY@tc##1{\textcolor[rgb]{0.60,0.60,0.60}{##1}}}
\expandafter\def\csname PY@tok@na\endcsname{\def\PY@tc##1{\textcolor[rgb]{0.49,0.56,0.16}{##1}}}
\expandafter\def\csname PY@tok@nt\endcsname{\let\PY@bf=\textbf\def\PY@tc##1{\textcolor[rgb]{0.00,0.50,0.00}{##1}}}
\expandafter\def\csname PY@tok@nd\endcsname{\def\PY@tc##1{\textcolor[rgb]{0.67,0.13,1.00}{##1}}}
\expandafter\def\csname PY@tok@s\endcsname{\def\PY@tc##1{\textcolor[rgb]{0.73,0.13,0.13}{##1}}}
\expandafter\def\csname PY@tok@sd\endcsname{\let\PY@it=\textit\def\PY@tc##1{\textcolor[rgb]{0.73,0.13,0.13}{##1}}}
\expandafter\def\csname PY@tok@si\endcsname{\let\PY@bf=\textbf\def\PY@tc##1{\textcolor[rgb]{0.73,0.40,0.53}{##1}}}
\expandafter\def\csname PY@tok@se\endcsname{\let\PY@bf=\textbf\def\PY@tc##1{\textcolor[rgb]{0.73,0.40,0.13}{##1}}}
\expandafter\def\csname PY@tok@sr\endcsname{\def\PY@tc##1{\textcolor[rgb]{0.73,0.40,0.53}{##1}}}
\expandafter\def\csname PY@tok@ss\endcsname{\def\PY@tc##1{\textcolor[rgb]{0.10,0.09,0.49}{##1}}}
\expandafter\def\csname PY@tok@sx\endcsname{\def\PY@tc##1{\textcolor[rgb]{0.00,0.50,0.00}{##1}}}
\expandafter\def\csname PY@tok@m\endcsname{\def\PY@tc##1{\textcolor[rgb]{0.40,0.40,0.40}{##1}}}
\expandafter\def\csname PY@tok@gh\endcsname{\let\PY@bf=\textbf\def\PY@tc##1{\textcolor[rgb]{0.00,0.00,0.50}{##1}}}
\expandafter\def\csname PY@tok@gu\endcsname{\let\PY@bf=\textbf\def\PY@tc##1{\textcolor[rgb]{0.50,0.00,0.50}{##1}}}
\expandafter\def\csname PY@tok@gd\endcsname{\def\PY@tc##1{\textcolor[rgb]{0.63,0.00,0.00}{##1}}}
\expandafter\def\csname PY@tok@gi\endcsname{\def\PY@tc##1{\textcolor[rgb]{0.00,0.63,0.00}{##1}}}
\expandafter\def\csname PY@tok@gr\endcsname{\def\PY@tc##1{\textcolor[rgb]{1.00,0.00,0.00}{##1}}}
\expandafter\def\csname PY@tok@ge\endcsname{\let\PY@it=\textit}
\expandafter\def\csname PY@tok@gs\endcsname{\let\PY@bf=\textbf}
\expandafter\def\csname PY@tok@gp\endcsname{\let\PY@bf=\textbf\def\PY@tc##1{\textcolor[rgb]{0.00,0.00,0.50}{##1}}}
\expandafter\def\csname PY@tok@go\endcsname{\def\PY@tc##1{\textcolor[rgb]{0.53,0.53,0.53}{##1}}}
\expandafter\def\csname PY@tok@gt\endcsname{\def\PY@tc##1{\textcolor[rgb]{0.00,0.27,0.87}{##1}}}
\expandafter\def\csname PY@tok@err\endcsname{\def\PY@bc##1{\setlength{\fboxsep}{0pt}\fcolorbox[rgb]{1.00,0.00,0.00}{1,1,1}{\strut ##1}}}
\expandafter\def\csname PY@tok@kc\endcsname{\let\PY@bf=\textbf\def\PY@tc##1{\textcolor[rgb]{0.00,0.50,0.00}{##1}}}
\expandafter\def\csname PY@tok@kd\endcsname{\let\PY@bf=\textbf\def\PY@tc##1{\textcolor[rgb]{0.00,0.50,0.00}{##1}}}
\expandafter\def\csname PY@tok@kn\endcsname{\let\PY@bf=\textbf\def\PY@tc##1{\textcolor[rgb]{0.00,0.50,0.00}{##1}}}
\expandafter\def\csname PY@tok@kr\endcsname{\let\PY@bf=\textbf\def\PY@tc##1{\textcolor[rgb]{0.00,0.50,0.00}{##1}}}
\expandafter\def\csname PY@tok@bp\endcsname{\def\PY@tc##1{\textcolor[rgb]{0.00,0.50,0.00}{##1}}}
\expandafter\def\csname PY@tok@fm\endcsname{\def\PY@tc##1{\textcolor[rgb]{0.00,0.00,1.00}{##1}}}
\expandafter\def\csname PY@tok@vc\endcsname{\def\PY@tc##1{\textcolor[rgb]{0.10,0.09,0.49}{##1}}}
\expandafter\def\csname PY@tok@vg\endcsname{\def\PY@tc##1{\textcolor[rgb]{0.10,0.09,0.49}{##1}}}
\expandafter\def\csname PY@tok@vi\endcsname{\def\PY@tc##1{\textcolor[rgb]{0.10,0.09,0.49}{##1}}}
\expandafter\def\csname PY@tok@vm\endcsname{\def\PY@tc##1{\textcolor[rgb]{0.10,0.09,0.49}{##1}}}
\expandafter\def\csname PY@tok@sa\endcsname{\def\PY@tc##1{\textcolor[rgb]{0.73,0.13,0.13}{##1}}}
\expandafter\def\csname PY@tok@sb\endcsname{\def\PY@tc##1{\textcolor[rgb]{0.73,0.13,0.13}{##1}}}
\expandafter\def\csname PY@tok@sc\endcsname{\def\PY@tc##1{\textcolor[rgb]{0.73,0.13,0.13}{##1}}}
\expandafter\def\csname PY@tok@dl\endcsname{\def\PY@tc##1{\textcolor[rgb]{0.73,0.13,0.13}{##1}}}
\expandafter\def\csname PY@tok@s2\endcsname{\def\PY@tc##1{\textcolor[rgb]{0.73,0.13,0.13}{##1}}}
\expandafter\def\csname PY@tok@sh\endcsname{\def\PY@tc##1{\textcolor[rgb]{0.73,0.13,0.13}{##1}}}
\expandafter\def\csname PY@tok@s1\endcsname{\def\PY@tc##1{\textcolor[rgb]{0.73,0.13,0.13}{##1}}}
\expandafter\def\csname PY@tok@mb\endcsname{\def\PY@tc##1{\textcolor[rgb]{0.40,0.40,0.40}{##1}}}
\expandafter\def\csname PY@tok@mf\endcsname{\def\PY@tc##1{\textcolor[rgb]{0.40,0.40,0.40}{##1}}}
\expandafter\def\csname PY@tok@mh\endcsname{\def\PY@tc##1{\textcolor[rgb]{0.40,0.40,0.40}{##1}}}
\expandafter\def\csname PY@tok@mi\endcsname{\def\PY@tc##1{\textcolor[rgb]{0.40,0.40,0.40}{##1}}}
\expandafter\def\csname PY@tok@il\endcsname{\def\PY@tc##1{\textcolor[rgb]{0.40,0.40,0.40}{##1}}}
\expandafter\def\csname PY@tok@mo\endcsname{\def\PY@tc##1{\textcolor[rgb]{0.40,0.40,0.40}{##1}}}
\expandafter\def\csname PY@tok@ch\endcsname{\let\PY@it=\textit\def\PY@tc##1{\textcolor[rgb]{0.25,0.50,0.50}{##1}}}
\expandafter\def\csname PY@tok@cm\endcsname{\let\PY@it=\textit\def\PY@tc##1{\textcolor[rgb]{0.25,0.50,0.50}{##1}}}
\expandafter\def\csname PY@tok@cpf\endcsname{\let\PY@it=\textit\def\PY@tc##1{\textcolor[rgb]{0.25,0.50,0.50}{##1}}}
\expandafter\def\csname PY@tok@c1\endcsname{\let\PY@it=\textit\def\PY@tc##1{\textcolor[rgb]{0.25,0.50,0.50}{##1}}}
\expandafter\def\csname PY@tok@cs\endcsname{\let\PY@it=\textit\def\PY@tc##1{\textcolor[rgb]{0.25,0.50,0.50}{##1}}}

\def\PYZbs{\char`\\}
\def\PYZus{\char`\_}
\def\PYZob{\char`\{}
\def\PYZcb{\char`\}}
\def\PYZca{\char`\^}
\def\PYZam{\char`\&}
\def\PYZlt{\char`\<}
\def\PYZgt{\char`\>}
\def\PYZsh{\char`\#}
\def\PYZpc{\char`\%}
\def\PYZdl{\char`\$}
\def\PYZhy{\char`\-}
\def\PYZsq{\char`\'}
\def\PYZdq{\char`\"}
\def\PYZti{\char`\~}
% for compatibility with earlier versions
\def\PYZat{@}
\def\PYZlb{[}
\def\PYZrb{]}
\makeatother


    % Exact colors from NB
    \definecolor{incolor}{rgb}{0.0, 0.0, 0.5}
    \definecolor{outcolor}{rgb}{0.545, 0.0, 0.0}



    
    % Prevent overflowing lines due to hard-to-break entities
    \sloppy 
    % Setup hyperref package
    \hypersetup{
      breaklinks=true,  % so long urls are correctly broken across lines
      colorlinks=true,
      urlcolor=urlcolor,
      linkcolor=linkcolor,
      citecolor=citecolor,
      }
    % Slightly bigger margins than the latex defaults
    
    \geometry{verbose,tmargin=1in,bmargin=1in,lmargin=1in,rmargin=1in}
    
    

    \begin{document}
    
    
    \maketitle
    
    

    
    \subsection{NCX mice model}\label{ncx-mice-model}

    \subsubsection{Equations}\label{equations}

    \subsubsection{Voltage activated Na
current}\label{voltage-activated-na-current}

This is the sodium curent,calculated from a single (file ?) channel with
constant field formulation of current

    \(\large{k_1 = 0.025 \times e^{\frac{em+90}{12}}}\)

\(\large{f_o = f_o + (f_rk_1 - f_o(\frac{1}{2} + \frac{1}{4k_1}))dt}\)

\(\large{f_r = f_r + ((1-f_r)\frac{0.15}{k_1} - f_rk_1)dt}\)

\(\large{k_{em1} = e^{\frac{em}{26}}}\)

\(\large{ i_{na} = (-600~f_o^{2})~em~\frac{(n_o -n_iK_{em1})}{(100 + n_o + n_i)(K_{em1} -1.0)} }\)

    \begin{center}\rule{0.5\linewidth}{\linethickness}\end{center}

    \begin{verbatim}
k1=0.025*exp((em+90)/12);
fo=fo+ (fr*k1-fo*(0.5+0.25/k1))*dt;
fr=fr+((1-fr)*0.15/k1 - fr*k1)*dt;
kem1 = exp( em / 26.0);
ina =(-600* fo^2 )* em * (no  - ni * kem1) /( (100 + no + ni)*(kem1 - 1.0 ));
\end{verbatim}

    \begin{center}\rule{0.5\linewidth}{\linethickness}\end{center}

    \begin{center}\rule{0.5\linewidth}{\linethickness}\end{center}

    \subsubsection{Voltage activated Ca
current}\label{voltage-activated-ca-current}

This is voltage activated Ca current

    \(\large{k_{ca} = 0.025 e^{\frac{(em+60)}{12}}}\)

\(\large{f_{oc} = f_{oc} +(f_{rc} k_{ca}-f_{oc}(\frac{1}{2}+\frac{1}{k_{ca}}))dt}\)

\(\large{f_{rc} = f_{rc} +((1-f_{rc})\frac{0.15}{k{ca}}-f_{r}k_{ca})dt}\)

\(\large{i_{ca} = (-900~f_o^{2}~em~\frac{(c_o-c_ik_{em1}^2)}{(100+c_o+c_i)(k_{em1}^2-1.0)}}\)

    \begin{verbatim}
kca=0.025*exp((em+60)/12);
foc=foc+ (frc*kca-foc*(0.5+0.25/kca))*dt;
frc=frc+((1-frc)*0.15/kca - fr*kca)*dt;
ica= -900* fo^2 * em * (co  - ci * kem1^2) /( (100 + co + ci)*(kem1^2 - 1.0 ));
\end{verbatim}

    \begin{center}\rule{0.5\linewidth}{\linethickness}\end{center}

    \subsubsection{Open probability of inward rectifier K
channel}\label{open-probability-of-inward-rectifier-k-channel}

This is an inward rectifier K channel probability of being open

    \(\large{f_{irk}} = \frac{1}{1+e^\frac{(em+60)}{15}}\)

    \begin{verbatim}
firk=1/(1+exp((em+60)/15));
\end{verbatim}

    \subsubsection{Open probability of a delayed rectifier K
channel}\label{open-probability-of-a-delayed-rectifier-k-channel}

This is calculation of the opening probability of a delayed K channel

    \(\large k_{edk} = e^\frac{(em+42)}{10}\)

\(\large f_{dk1} = f_{dk1}+ ((1-f_{dk1}-f_{dk2}-f_{dk3})\times k_{edk}+\frac{f_{dk2}}{k_{edk}}-f_{dk1}\times(k_{edk}+\frac{15}{k_{edk}})) \times 0.0003dt\)

\(\large f_{dk2} = f_{dk2}+(f_{dk1}\times k_{edk} - 15 \times \frac{f{dk2}}{kedk}) \times 0.0003 dt\)

\(\large f_{dk3} = f_{dk3}+ (f_{dk2} \times k_{edk}-15 \times \frac{f{dk3}}{k{edk}}) \times 0.0003dt\)

    \begin{verbatim}
kedk=exp((em+42)/10);
fdk1=fdk1+((1-fdk1-fdk2-fdk3)*kedk + fdk2/kedk  -fdk1*(kedk+15/kedk))*0.0003*dt;
fdk2=  fdk2 + (fdk1*kedk-15*fdk2/kedk)*0.0003*dt;
fdk3=  fdk3 + (fdk2*kedk-15*fdk3/kedk)*0.0003*dt;
\end{verbatim}

    \subsubsection{Total K current}\label{total-k-current}

Which is the sum of inward rectifier and delayed K current. Like Na
channel, the current is caculated from a constant field equation for a
single file channel

    \(\large i_k = -(250 \times f_{irk}+ f_{dk3} \times 220) \times em \times \frac{(k_o-k_ik_{em1})}{((50+k_o+k_i)(k_{em1} - 1.0))}\)

    \texttt{ik\ =-\ (250*firk+fdk3*220)\ *\ em\ *\ (ko\ \ -\ ki\ *\ kem1)\ /(\ (50\ +\ ko\ +\ ki)*(kem1\ -\ 1.0\ ));}

    \subsubsection{Na/K pump function}\label{nak-pump-function}

This is a primitive Na/K pump function which is assumed to be just
proportional to a hill equation with 3 Na binding.

    \$\large i\_\{pump\} = \frac{n_i^3}{(ni^3+20^3)} \times 6 \times 200 \$

    \begin{verbatim}
ipump=ni^3/(ni^3+20^3)*6*200;
\end{verbatim}

    \subsection{Na/Ca exchange current}\label{naca-exchange-current}

This is the primitive Na/Ca exchange system employe.Two Na compete with
1 Ca and 1 Na binding independently from Ca, similar to what John Reeves
found 40 years ago.

    \(\large d_{out} = 1+ \frac{c_o}{0.01}+ \frac{no}{20(1+\frac{n_o}{20})}\)

\(\large d_{in} = 1+ \frac{c_i}{0.01}+ \frac{ni}{20(1+\frac{n_i}{20})}\)

\(\large f_{co} = \frac{c_o}{0.1 \times d_{out}}\)

\(\large f_{2no} = \frac{n_o \times n_o}{20 \times 20 \times d_{out}}\)

\(\large f_{ci} = \frac{ci}{0.01 \times d_{in}}\)

\(\large f_{2ni} = \frac{n_i \times n_i}{20 \times 20 \times d_{out}}\)

\(\large f_{3ni} = \frac{f_{2ni} \times n_i}{(n_i+30)}\)

\(\large f_{3no} = \frac{f_{2no} \times n_o}{(n_o+30)}\)

\(\large k_{em} = e^\frac{em}{55}\)

\(\large i_{ncx} = \frac{80(f_{co} \times f_{3ni} \times k_{em} - f_{ci}\frac{f_{3no}}{k_{em}})}{f_{co}+f_{3ni} \times k_{em}+f_{ci}+\frac{f_{3no}}{K_{em}}}\)

    \begin{verbatim}
dout=1+co/0.01+no/20*(1+no/20);
din=1+ci/0.01+ni/20*(1+ni/20);
fco=co/0.01/dout;
f2no=no*no/20/20/dout;
fci=ci/0.01/din;
f2ni=ni*ni/20/20/din;
f3ni=f2ni*ni/(ni+30);
f3no=f2no*no/(no+30);
kem=exp(em/55);
incx=80*(fco*f3ni*kem-fci*f3no/kem)/(fco+f3ni*kem+fci+f3no/kem);
\end{verbatim}

    \subsubsection{SR Ca pump function}\label{sr-ca-pump-function}

This is the SR Ca pump function as it depends on binding Ca on the
cytoplasmic side (fcain) and in the SR lumen (fcasr)

    \(\large f_{cain} = \frac{c_i}{c_i+0.002}\)

\(\large f_{casr} = \frac{casr}{casr+2}\)

\(\large f_{rel} = -\frac{i_{ca}}{(-i_{ca}+100)}\)

\(\large f_{srinact} = f_{srinact}+((1-f_{srinact}) \times f_{rel} \times 2-f_{srinact} \times 0.005)dt\)

\(\large d_{casr} = (0.015 \times f_{cain}-0.002 \times f_{casr})-f_{rel}\times (1-f_{srinact}) \times casr \times 0.13\)

\(\large {casr} = (casr +d_{casr}10)dt\)

    \begin{verbatim}
fcain=ci/(ci+0.002);
fcasr=casr/(casr+2);
frel=-ica/(-ica+100);
fsrinact=fsrinact+((1-fsrinact)*frel*2-fsrinact*0.005)*dt; 
dcasr=(0.015*fcain-0.002*fcasr)-frel*(1-fsrinact)*casr*0.13;
casr=casr+dcasr*10*dt;
\end{verbatim}

    \subsubsection{This is the cytoplasmic Na
concentration}\label{this-is-the-cytoplasmic-na-concentration}

    \(\large {n_i} = n_{i}-(i_{na}+i_{pump} \times 3+i_{ncx} \times 3) \times 10^-6 \times dt\)

    \texttt{ni=ni-(ina+ipump*3+incx*3)*10\^{}-6*dt\ ;}

    \subsubsection{This is the cytoplasmic K
concentration}\label{this-is-the-cytoplasmic-k-concentration}

    \(\large k_i = k_i - (i_k-i_{pump} \times 2) \times 10^-6 \times dt\)

    \texttt{ki=ki-(ik-ipump*2)*10\^{}-6*dt;}

    \subsubsection{This is the total cytoplasmic Ca
concentration}\label{this-is-the-total-cytoplasmic-ca-concentration}

And the free Ca concentration, assumed to be 40 times less

    \(\large c_{itot} = c_{itot}-((\frac{i_{ica}}{2}-i_{ncx}) \times 10^-6 + d_{casr}) \times dt\)

\(\large c_i = \frac{c_{itot}}{40}\)

    \begin{Verbatim}[commandchars=\\\{\}]
{\color{incolor}In [{\color{incolor} }]:} \PY{n}{citot}\PY{o}{=}\PY{n}{citot}\PY{o}{\PYZhy{}}\PY{p}{(}\PY{p}{(}\PY{n}{ica}\PY{o}{/}\PY{l+m+mi}{2}\PY{o}{\PYZhy{}}\PY{n}{incx}\PY{p}{)}\PY{o}{*}\PY{l+m+mi}{10}\PY{o}{\PYZca{}}\PY{o}{\PYZhy{}}\PY{l+m+mi}{6} \PY{o}{+}\PY{n}{dcasr}\PY{p}{)}\PY{o}{*}\PY{n}{dt}\PY{p}{;}
        \PY{n}{ci}\PY{o}{=}\PY{n}{citot}\PY{o}{/}\PY{l+m+mi}{40}\PY{p}{;}
\end{Verbatim}


    \subsubsection{Calculation of the membrane
potentail}\label{calculation-of-the-membrane-potentail}

This is the membrane potential calculated from total charges in the
cytoplasm. The number, 120000 converts charge excess to Em and reflects
the capacitance of the cell

    \(\large em = (n_i+k_i+2 \times c_{itot}+ 2 \times\frac{casr}{10}-anion) \times 12000\)

    \texttt{em=(ni+ki+2*citot+2*casr/10-anion)*12000;}


    % Add a bibliography block to the postdoc
    
    
    
    \end{document}
